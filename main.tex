% main.tex
% 代数幾何学の文書ファイル本体

% 作成したカスタムクラスファイルを指定
\documentclass{myclass}

% 作成したカスタムスタイルファイルを読み込み
\usepackage{mystyle}

% --- 文書情報 ---
\title{代数幾何学}
\author{Fefr}
\date{\today}

% ==================================================
% --- コンパイル手順の注意 ---
% この文書を正しくPDF化するには、複数回のコンパイルが必要です。
% VScodeのlatex-workshopなどをお使いの場合、以下の順で実行されるレシピを作成してください。
%
% 1. ptex2pdf -l main.tex  (本文のコンパイル、.aux, .idx ファイルなどを生成)
% 2. mendex -s myindex.ist main.idx (索引の作成、.idx -> .ind)
% 3. pbibtex main (参考文献の処理、.aux -> .bbl)
% 4. ptex2pdf -l main.tex  (索引と参考文献を本文に反映)
% 5. ptex2pdf -l main.tex  (相互参照を確定させるため、もう一度実行)
% ==================================================


\setlength\parindent{0pt}

\begin{document}

% --- タイトルページ ---
\maketitle

% --- 目次 ---
\tableofcontents

% ==================================================
% --- 前付 (まえがきなど) ---
\frontmatter

\chapter*{まえがき}
本書は古典的な代数幾何について論ずる.主に\cite{Nakano}を参考にするつもりである.


\cleardoublepage

% ==================================================
% --- 本文 ---
\mainmatter
\chapter{代数多様体}
この章では代数幾何の対象である"代数多様体(algebraic variety)"を定義する.

\section{アフィン空間}
アフィン空間を復習する.\\
\begin{definition}
  $\A$を空でない集合.$V$を$\C$上の有限次元ベクトル空間とする.次の条件を満たす写像の族$\{T_v:\A \to \A :v\in V\}$が与えられているとき,組$(\A,V)$あるいは単に$\A$を\textbf{アフィン空間}\index{アフィンくうかん@アフィン空間}という.
  \begin{enumerate}
    \item 任意の$v,w\in V$に対して,$T_{v+w} = T_v \circ T_w$が成り立つ.
    \item 任意の$P,Q\in \A$に対して,ただ一つの$v\in V$が存在して,$T_v(P) = Q$となる.
  \end{enumerate}
\end{definition}

\begin{definition}
  アフィン空間$\A$の次元はそれに伴うベクトル空間$V$の次元$\dim V$で定める.
\end{definition}


次は容易にわかる.

\begin{proposition}
  $(\A,V)$をアフィン空間とすると,次が成り立つ.
  \begin{enumerate}
    \item $0\in V$に対応する写像$T_0$は恒等写像$1_\A$である.
    \item 任意の$v\in V$に対して,$T_v$は全単射.
  \end{enumerate}
\end{proposition}


\begin{definition}
  定義の2の$v$を$\overrightarrow{PQ}$とかく\footnote{従って,$-v$に対応するのは$\overrightarrow{QP}$である.}.また,$T_v(P)$を$P+v$とかく\footnote{イメージは$T_v$は$v$だけ平行移動を行う写像である.}.
\end{definition}

\begin{proposition}
  $(\A,V)$をアフィン空間とすると,任意の$P,Q,R\in \A$に対して,次が成り立つ.
  \begin{equation*}
    \overrightarrow{PQ} + \overrightarrow{QR} = \overrightarrow{PR}
  \end{equation*}
\end{proposition}
\begin{proof}
  $Q = P + \overrightarrow{PQ}$,$R = Q + \overrightarrow{QR}$より,
  \begin{equation*}
    R = P + (\overrightarrow{PQ} + \overrightarrow{QR})
  \end{equation*}
  となる.一方,$R = P + \overrightarrow{PR}$であるから,$\overrightarrow{PR} = \overrightarrow{PQ} + \overrightarrow{QR}$が従う.
\end{proof}

有限次元ベクトル空間$V$に座標系$V \stackrel{\cong}{\longrightarrow} \C^n$を入れるように,アフィン空間にも座標系を入れることができる.

\begin{definition}
  $\A$をアフィン空間とし,$\A$の点$O$と$V$の基底$\{e_1,\cdots,e_n\}$をとる.$\A$の任意の点$P$
  に対して,$\overrightarrow{OP}\in V$が定まり,$V$の基底を用いて,$\overrightarrow{OP} = \sum_i P_ie_i$とできる.このときの対応
  \begin{equation*}
    \begin{array}{rcl}
      \A & \to & \C^n \\
      P & \mapsto & (P_1,\cdots,P_n)
    \end{array}
  \end{equation*}
  を$\A$の\textbf{(アフィン)座標系}\index{アフィンざひょうけい@アフィン座標系}といい,$(O;e_1,\cdots,e_n)$とかく\footnote{アフィン座標系$X:\A \to \C^n$は各$i$成分への射影$\pi_i$と合成し$X_i:=\pi_i \circ X:\A\to \C$が定義できることに注意する.}.
\end{definition}

\begin{remark}
  座標系を導入することで$n$次元アフィン空間$\A$の点は$\C^n$の点と一対一に対応する.従って,誤解のおそれがない場合$\C^n$を$n$次元アフィン空間ということがある.
\end{remark}


\begin{proposition}
  アフィン空間$\A$に対して,二つの異なる座標系$(O;e_1,\cdots,e_n)$,$(O';e_1',\cdots,e_n')$が与えられたとする.このとき,ある$A\in \GL(n,\C)$と$b\in \C^n$が存在して,任意の$P\in \A$に対して,
  \begin{equation*}
    \begin{pmatrix}
      P_1' \\ \vdots \\ P_n'
    \end{pmatrix}
    =
    A
    \begin{pmatrix}
      P_1 \\ \vdots \\ P_n
    \end{pmatrix}
    +
    b
  \end{equation*}
  が成り立つ.
\end{proposition}

\begin{proof}
  証明はベクトル空間における基底変換の議論と同様である.
\end{proof}

\begin{definition}
  先程の命題における変換$P' = AP + b$を\textbf{アフィン変換}\index{アフィンへんかん@アフィン変換}という.
\end{definition}



\section{アフィン空間内の代数的集合}
ここで,可微分多様体の定義を思い出すと,それは\\

$\C$上の$n$次元アフィン空間を$\A$(または次元を明示して$\A^n$)とかく.
一組のアフィン座標$X_1,\cdots,X_n$をとると,$\A$の点$p$は座標$(X_1(p),\cdots,X_n(p))$で表される.
\begin{definition}
  $f:\A \to \C$が\textbf{アフィン関数}\index{アフィンかんすう@アフィン関数}とは,多項式$F(X_1,\cdots,X_n)$があって,すべての$p\in \A$に対して,
  \begin{equation*}
    f(p) = F(X_1(p),\cdots,X_n(p))
  \end{equation*}
  となってるいるときを言う.アフィン関数という概念はアフィン座標の取り方によらず定まる\footnote{つまり,アフィン変換で不変ということである.}.\\
  $\A$上のアフィン関数全体$\C[\A]$は自然に環.さらに$\C$-代数になる.$f\mapsto F$という対応によって,同型
  \begin{equation*}
    \C[\A] \cong \C[X_1,\cdots,X_n]
  \end{equation*}
  が誘導される.$\A$に$\C[\A]$を付随させて考えるとき,代数多様体としてのアフィン空間という.以下この意味でアフィン空間という言葉を用いる.
\end{definition}




\begin{definition}
  $\A$の部分集合$V$が\textbf{(アフィン)代数的集合}\index{アフィンだいすうてきしゅうごう@アフィン代数的集合}であるとは,$\C[\A]$の元$f_1,\cdots,f_r$
  があって,
  \begin{equation*}
    V = \{p \in \A : f_1(p) = \cdots = f_r(p) = 0\}
  \end{equation*}
  となるときをいう.
\end{definition}
このとき,$V$は閉集合である.$\C[\A]$の関数の$V$への制限を$V$上のアフィン関数という.その全体$\C[V]$も同様に$\C$-代数となる.
以降代数的集合$V$というときは,$\C[V]$を付随させて考えるものとする.\\
制限写像
\begin{equation*}
  \C[\A] \to \C[V];f\mapsto f|_V
\end{equation*}
は$\C$-代数の準同型であり,その定め方から全射となる.その核を$I(V)$と書くことにすれば,
\begin{align*}
  &I(V) = \{f\in \C[\A] : f|_V = 0\}\\
  &\C[V] \cong \C[\A]/I(V)
\end{align*}
となる.この$I(V)$を$V$の\textbf{定義イデアル}\index{ていぎイデアル@定義イデアル}という.\\
次のことは基本的である.

\begin{proposition}
  $V_1,V_2$を代数的集合とすると,以下が成り立つ.
  \begin{enumerate}
    \item $V = \{p \in \A : \forall f \in I(V),f(p) = 0\}$
    \item $V_1\subset V_2 \Leftrightarrow I(V_2)\subset I(V_1)$
    \item $\C[\A]$の元$f$に対して,自然数$n$があって$f^n\in I(V)$ならば$f \in I(V)$となる\footnote{つまり,$\sqrt{I(V)} = I(V)$である.}.
  \end{enumerate}
\end{proposition}

\begin{definition}
  $\C[\A]$のイデアル$I$に対して,その\textbf{零点集合}\index{れいてんしゅうごう@零点集合}$V(I)$は次のように定義される.
  \begin{equation*}
    V(I) = \{p\in \A : \forall f \in I, f(p) = 0\}
  \end{equation*}
  先程の命題の1は$V=V(I(V))$とかける.
\end{definition}

\begin{proposition}
  代数的集合の定義イデアルは有限個の素イデアルの共通部分である.逆に$\C[\A]$の素イデアルの共通部分はある代数的集合の定義イデアルである.
\end{proposition}

\begin{proof}
  代数的集合$V$の定義イデアル$I(V)$は,ネーター環$\C[\A]$のイデアルだから,有限個の準素イデアルの共通部分として表される:
  \begin{equation*}
    I(V) = \mathfrak{q}_1\cap \cdots\cap \mathfrak{q}_r
  \end{equation*}
  $\mathfrak{q}_j$が準素イデアルなのでその根基$\sqrt{\mathfrak{q}_j}$は素イデアルで,$\mathfrak{q}_j\subset \sqrt{\mathfrak{q}_j}$なので
  \begin{equation*}
    I(V) \subset \sqrt{\mathfrak{q}_1}\cap \cdots \cap \sqrt{\mathfrak{q}_r}\qquad \cdots (*)
  \end{equation*}
  また,右辺の元$f$をとると,ある自然数$n_j$があって,$f^{n_j} \in \mathfrak{q}_j$であり,$n = \max_j n_j$とすると,
  \begin{equation*}
    f^n \in \mathfrak{q}_1\cap \cdots \cap \mathfrak{q}_r = I(V)
  \end{equation*}
  よって,$f\in I(V)$となり,$(*)$の等号が成り立つ.逆は読者に任せる.
\end{proof}

\begin{remark}
  定義イデアル$I(V)$の素イデアルへの分解$I(V) = \bigcap_k \mathfrak{p}_k$の$\mathfrak{p}_i$は他の素イデアルとの包含関係がないとして取れる. 
\end{remark}


\section{アフィン代数多様体}

\begin{definition}
  アフィン空間$\A$内の代数的集合$V$が\textbf{既約}\index{きやく@既約}とは,任意の代数的集合$V_1,V_2$に対して$V = V_1\cup V_2$ならば$V=V_1$または$V=V_2$が成り立つときをいう.
\end{definition}

\begin{proposition}
  $V$が既約$\Leftrightarrow$$I(V)$が素イデアル
\end{proposition}
\begin{proof}
  $(\Rightarrow)$
  $I(V) = \mathfrak{p}_1\cap \cdots \cap \mathfrak{p}_r$が$I(V)$の最短表示であって,$r\geq 2$とすると,
  \begin{equation*}
    V(\mathfrak{p}_j) \not \subset \bigcup_{j\neq i}V(\mathfrak{p}_i)
  \end{equation*}
  であることに注意すれば,$V$が既約であることに矛盾する.よって$r=1$.これは$I(V)$が素イデアルであることを示している.\\
  $(\Leftarrow)$$V=V_1\cup V_2$で,$V_1,V_2\neq V$とすると$I(V) = I(V_1)\cap I(V_2) = (\mathfrak{p}_1\cap \cdots \cap \mathfrak{p}_r)\cap (\mathfrak{p}'_1 \cap \cdots \cap \mathfrak{p}'_s)$とかける.
  ここで,素イデアルの集合$\{\mathfrak{p}_i,\mathfrak{p}'_j\}_{i,j}$には少なくとも二つの極小元があることに注意する.実際,一つならば$I(V) = I(V_1)$か$I(V) = I(V_2)$が成り立つのでおかしい.
  これは,$I(V)$が素イデアルであることに矛盾する.
\end{proof}



% ==================================================
% --- 後付 (索引、参考文献) ---
\backmatter

% --- 索引の出力 ---
\printindex

% --- 参考文献の出力 ---
% スタイルファイルはjunsrtを指定
\bibliographystyle{junsrt-j}
% .bibファイルを指定
\bibliography{mybib}


\end{document}
